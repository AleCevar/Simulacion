\documentclass{beamer}
\usepackage{graphicx} % Required for inserting images

\usepackage{pgfplots}
\pgfplotsset{compat=1.18}
\usepackage{amsmath}

\title[Pruebas Estadísticas]{Pruebas Estadísticas}
\subtitle{Tarea III}
\author{Alejandro Cerdas \\ Kener Castillo}
\institute{Curso: Simulación \\ Profesor: Eddy Ramírez Jiménez}
\date{\today}

\usetheme{Warsaw}
\usecolortheme{orchid}

\newcommand{\testZTwo}[4]{% #1=l #2=r #3=zobs
\begin{tikzpicture}
\begin{axis}[
  domain=-4:4, samples=200,
  axis lines=left,
  xlabel={$z$}, ylabel={Densidad},
  title={#4},
  ytick=\empty,
  xtick={#1,#2},
  xticklabels={$#1$,$#2$},
  tick label style={/pgf/number format/fixed}
]
\addplot[thick] {1/sqrt(2*pi)*exp(-x^2/2)};

% regiones críticas
\addplot[domain=-4:#1, fill=red, fill opacity=0.3]
  {1/sqrt(2*pi)*exp(-x^2/2)} \closedcycle;
\addplot[domain=#2:4, fill=red, fill opacity=0.3]
  {1/sqrt(2*pi)*exp(-x^2/2)} \closedcycle;

% líneas críticas
\addplot[dashed, thick] coordinates {(#1,0) (#1,0.4)};
\addplot[dashed, thick] coordinates {(#2,0) (#2,0.4)};

% observado
\addplot[very thick] coordinates {(#3,0) (#3,0.35)};
\node[anchor=north] at (axis cs:#3,0) {$z_{obs}=#3$};

\end{axis}
\end{tikzpicture}
}

\begin{document}

\begin{frame}
    \titlepage
\end{frame}

\begin{frame}{Introducción}
    Las pruebas estadísticas que se usaron fueron las mismas vistas en clase:
    \begin{itemize}
        \item Promedio
        \item Varianza
        \item Corridas por incrementos (continuos) y arriba/abajo (discretos)
        \item Dígitos
        \item Números
        \item Poker
        \item Series
    \end{itemize}
    \vspace{0.5cm}
    \centering
    \begin{tabular}{c c c c c c c}
        \hline
         & C & Python & Erlang & Racket & Java & C++ \\
        \hline
        Continuos & [1,4] y [1,8] & [1,6] & NA & [1,20] & NA & [1,10]\\
        Discretos & NA & [0,1[ & [0,1[ & NA & [0,1[ & NA\\
        \hline
    \end{tabular}
\end{frame}

\begin{frame}{Fórmulas}
    \begin{block} {Prueba de Promedio y Varianza}
        \textbf{Discretos:} \\
        \vspace{0.1cm}
        $\sigma^2 = \frac{(r-l+1)^2 - 1}{12}$ \\
        $\mu = \frac{\sum_{i=l}^r i}{r-l+1}$ 
    \end{block}
    \begin{block} {Prueba de Corrida Arriba/Abajo}
        \textbf{Discretos:} \\
        \vspace{0.1cm}
        $E(h) = \frac{2 \times m \times n}{n + m + 1}$ \\
        $\sigma ^ 2 = \frac{2 \times m \times n \times (2 \times m \times n - (m+n))}{(m+n)^2 \times (m+n-1)}$
    \end{block}
    \begin{block}{Prueba de Dígitos}
        \textbf{Discretos:}\\
        \vspace{0.1cm}
        $tamaño\ del\ hueco = base - 1$
    \end{block}
\end{frame}

\begin{frame}{Fórmulas}
    \begin{block} {Prueba de Números}
        \textbf{Discretos:}\\
        \vspace{0.1cm}
        $[3, min(base, 3 + \lfloor \frac{base}{3} \rfloor)]$
    \end{block}
    \begin{block} {Prueba de Poker}
        $P(diferentes) = \frac{\prod_{i = 0}^4 base - i}{base^5}$ \\
        $P(1\ pareja) = \frac{\prod_{i=0}^3 base - i}{base ^ 5} \times \binom{5}{2}$\\
        $P(2\ parejas) = \frac{\prod_{i=0}^2 base - i}{2 \times base^5} \times \binom{5}{2} \times \binom{3}{2}$\\
        $P(trio) = \frac{\prod_{i=0}^2 base - i}{base^5} \times \binom{5 }{3}$\\
        $P(full\ house) = \frac{\prod_{i=0}^1 base - i}{base^5} \times \binom{5}{3}$\\
        $P(poker) = \frac{\prod_{i=0}^1 base - i}{base^5} \times \binom{5}{4}$\\
        $P(quitilla) = \frac{base}{base^5}$
    \end{block}
\end{frame}

\begin{frame}{Fórmulas}
    \begin{block}{Prueba de Series}
        \textbf{Discretos:}\\
        \vspace{0.1cm}
        $cantidad\ de\ intervalos = \frac{base}{2}$\\
        $intervalos = [0,2,4,...,2*i,...,2*cantidad]$
    \end{block}
\end{frame}

\begin{frame}{Pruebas de Promedio}

\begin{columns}
\column{0.5\textwidth}
\centering
\resizebox{1.1\linewidth}{!}{
\testZTwo{-1.96}{1.96}{0.28}{Java $[0, 1[$}
}
\column{0.5\textwidth}
\centering
\resizebox{1.1\linewidth}{!}{
\testZTwo{-1.96}{1.96}{-0.05}{Erlang $[0, 1[$}
}
\end{columns}

\end{frame}

\begin{frame}{Pruebas de Promedio}

\begin{columns}
\column{0.5\textwidth}
\centering
\resizebox{1.1\linewidth}{!}{
\testZTwo{-1.96}{1.96}{1.12}{Python $[0, 1[$}
}
\column{0.5\textwidth}
\centering
\resizebox{1.1\linewidth}{!}{
\testZTwo{-1.96}{1.96}{-0.57}{Python $[1, 6]$}
}
\end{columns}
\end{frame}


\begin{frame}{Pruebas de Promedio}

\begin{columns}
\column{0.5\textwidth}
\centering
\resizebox{1.1\linewidth}{!}{
\testZTwo{-1.96}{1.96}{0.06}{C $[1, 4]$}
}
\column{0.5\textwidth}
\centering
\resizebox{1.1\linewidth}{!}{
\testZTwo{-1.96}{1.96}{0.68}{C $[1, 8]$}
}
\end{columns}
\end{frame}

\begin{frame}{Muestras Discretas}
    \begin{columns}
        \column{0.5\textwidth}
        \centering
        \includegraphics[0.5\textwidth]{1a4.png}
    \end{columns}
\end{frame}

\end{document}
